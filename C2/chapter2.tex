
In the field of ..... The rest of the chapter is organised as follows, section~\ref{chap2:rl_overview} introduces relevant concepts in \ac{RL}. Then, section~\ref{chap2:context_rl} explores how contextual information is involved in current state-of-the-art approaches. Followed by section~\ref{chap2:rl_robotics} that focuses on robotic applications, more concisely, on the use of \ac{RL} for the manipulation of rigid and deformable objects. Finally, the main findings of the literature review are discussed in section~\ref{chap2:discussion} and summarised in section~\ref{chap2:summary}.   

\section{Reinforcement learning overview}
\label{chap2:rl_overview}


\subsection{The Bellman optimality equation}

The Bellman optimality equation expresses the expected maximum or total reward that can be achieved from a given state based on the value function (Eq.~\eqref{C2eq:04}). The equation defines the relationship between the value of a state and the values of its neighbouring states. The Bellman optimality equation of the value function is given by the following:

\begin{equation}\label{C2eq:04}
V^{*}(s)=\max\limits_{a}\sum_{s'}^{}P(s'| s,a)(R(s,a,s')+\gamma V^{*}(s'))
\end{equation}


\section{Context and RL}
\label{chap2:context_rl}



\section{Reinforcement learning in robotics}
\label{chap2:rl_robotics}
This section introduces relevant research in the field of robotics and \ac{RL}, where classical and RL approaches are applied to solve \ac{HRI} problems and manipulation tasks of rigid and deformable objects.

\subsection{Why reinforcement learning in robotics?}
\label{chap2:rl_robotics_why}


\subsection{Robotic manipulation of rigid objects with reinforcement learning}
\label{chap2:rl_robotics_rigid}



\subsection{Reinforcement learning in human-robot interaction}
\label{chap2:rl_robotics_hri}


\subsection{Robotic manipulation of deformable objects with reinforcement learning}
\label{chap2:rl_robotics_deformable}

\section{Discussion}
\label{chap2:discussion}

This section delves into a detailed analysis of the advantages and disadvantages of ....

Additionally, current research endeavours have been made to explore the implementation of ...

Despite the advantages of the discussed approaches, there are still challenges related to ...

\begin{itemize}
    \item Lack of ...  
\end{itemize}


\section{Summary}
\label{chap2:summary}

This chapter has provided valuable insights into the implementation of ...

This literature review has also identified several gaps in the current knowledge of ....

To address these gaps, an incremental approach is followed in this thesis...