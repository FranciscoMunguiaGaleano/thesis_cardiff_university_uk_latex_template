\ac{AI} has significantly evolved over the years~\cite{shao2019survey}.....

\section{Motivation}

The performance of an \ac{AI} agent depends on...


To summarise, \ac{AI} performance relies on ...

\section{Research questions}

The following research questions will be addressed in this thesis:

\begin{itemize}
    \item How does ...?
    \item How to ...?
    \item Can ...?
    \item Is ...?
    \item Is \ac{AI} ...?
\end{itemize}

A detailed examination based on a literature review, experiments, and analysis will be conducted for each one of the questions above. These questions answers hold the potential to provide helpful information about the benefits and limitations of using affordances for training RL agents, as well as understanding the factors required to carry out ... 

\section{Aim and Objectives}

This thesis aims to investigate ..... To achieve this aim, the following objectives are pursued: 

\begin{itemize}
    \item To investigate ...
    \item To develop ...
    \item To develop ...
\end{itemize}

By pursuing these objectives, this thesis aims to contribute to ...

\section{Outline}

This outline presents a concise summary of the thesis' content comprised of six chapters, aiming to provide a clear overview of the research presented in the following chapters.\\


\noindent\textbf{Chapter \ref{sec:chapter2}} introduces relevant literature in ...\\

\noindent\textbf{Chapter \ref{sec:chapter3}}  introduces ....\\

\noindent\textbf{Chapter \ref{sec:chapter4}} introduces ...\\

\noindent\textbf{Chapter \ref{sec:chapter5}} presents .... \\

\noindent\textbf{Chapter \ref{sec:chapter6}} concludes the thesis by summarising the main contributions and achievements of this research work. In addition, the limitations and challenges of the approaches developed in this thesis are discussed, and propose future research directions.\\ 

\noindent Overall, this thesis aims to contribute to the ongoing efforts to improve ....